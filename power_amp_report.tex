%%%%%%%%%%%%%%%%%%%%%%%%%%%%%%%%%%%%%%%%%
% Lachaise Assignment
% LaTeX Template
% Version 1.0 (26/6/2018)
%
% This template originates from:
% http://www.LaTeXTemplates.com
%
% Authors:
% Marion Lachaise & François Févotte
% Vel (vel@LaTeXTemplates.com)
%
% License:
% CC BY-NC-SA 3.0 (http://creativecommons.org/licenses/by-nc-sa/3.0/)
% 
%%%%%%%%%%%%%%%%%%%%%%%%%%%%%%%%%%%%%%%%%

%----------------------------------------------------------------------------------------
%	PACKAGES AND OTHER DOCUMENT CONFIGURATIONS
%----------------------------------------------------------------------------------------

\documentclass{article}

\input{structure.tex} % Include the file specifying the document structure and custom commands

\usepackage{hyperref}

\usepackage{siunitx}

\usepackage{xcolor} % to color the conversation to look better

%----------------------------------------------------------------------------------------
%	ASSIGNMENT INFORMATION
%----------------------------------------------------------------------------------------

\title{Power Amplifier - SiFlower cn} % Title of the assignment

\author{Aleksandar Vuković} % Author name and email address

\date{\today} % University, school and/or department name(s) and a date

%----------------------------------------------------------------------------------------

\begin{document}

\maketitle % Print the title

%----------------------------------------------------------------------------------------
%	INTRODUCTION
%----------------------------------------------------------------------------------------

\section*{Introduction - Design Acceptance Criteria} % Unnumbered section

Below is the key specs of the PA for 5G band. 

\begin{itemize}
	\item Technology: TSMC 28nm 
	\item Supply voltage: 3.3V 
	\item Frequency range: from 4.9GHz to 5.9GHz 
	\item output power range: -25dBm to 0dBm with EVM below -38dB. 
	\item OIP3: 35dBm @0dBm % -10 to -12 dBm approx OP1dB 23 to 25 dBm
\end{itemize}

% When 23-25 dBm when translated to power for impedance
% 4.5 V - 5.6 V

\section{Questions}

Additional questions regarding design briefly presented:

\begin{itemize}
	\item input power, gain % i am not so sure about this question
	\item Is the input power range changing in the redesign? 
	\item gain control % for the highest output power expect the max gain setting (promised max output power)
	\item is power efficiency (PAE) of interest?
\end{itemize}

\subsection{Possible Design Considerations}

Design considerations that may be important and informative about the current design:

\begin{itemize}
	\item power combining technique - wilkinson or transformer % maybe seems to specific
	\item broadband power matching topology % to cover the 1 GHz range
	\item low output impedance for amplifier compared to 50 $\Omega$
\end{itemize}

\section{Papers to look at}

\begin{itemize}
	\item A 5.8 GHz class-AB power amplifier with 25.4 dBm saturation power and 29.7\% PAE
	\item A 5-5.8 GHz Fully-Integrated CMOS PA for WLAN Applications, Jeng-Han Tsai and Hong-Wun Ou-Yang
	\item Fully Integrated CMOS Power Amplifier With Efficiency Enhancement at Power Back-Off, Gang Liu, Peter Haldi, Tsu-Jae King Liu, Fellow, IEEE, and Ali M. Niknejad, Member, IEEE
\end{itemize}

Abstract from \href{https://ieeexplore.ieee.org/document/6830155}{A 5-5.8 GHz Fully-Integrated CMOS PA for WLAN Applications}, Jeng-Han Tsai and Hong-Wun Ou-Yang:

\begin{info}
	\textbf{Abstract} — A 5-5.8 GHz fully-integrated power amplifier is designed and fabricated in TSMC standard 0.18-$\mu$m 1P6M CMOS technology. Utilizing a two-way direct shunt combining technique with an odd mode suppression resistor, the CMOS PA achieves a measured maximum saturation output power (Psat) of 23.1 dBm at 5.2 GHz. The measured output 1-dB compression point (OP1dB) is 18.6 dBm and peak power-added efficiency (PAE) is 19.8 \% at 5.2 GHz. By using broadband power matching topology, the output power of the CMOS PA is 22.6 $\pm$ 0.5 dBm from 5 to 5.8 GHz. Index Terms — CMOS technology, radio frequency integrated circuits, power amplifiers, microwave amplifiers.
\end{info}

% a two-way direct shunt combining technique with an odd mode suppression resistor

% broadband power matching

Abstract from \href{https://www.sciengine.com/SCIS/doi/10.1007/s11432-016-0299-4?&trans=true}{A 5.8 GHz class-AB power amplifier with 25.4 dBm saturation power and 29.7\% PAE}, Chuan Qin,  Lei Zhang*,  Li Zhang,  Yan Wang,  Zhiping Yu

\begin{info}
	\textbf{Abstract} — In this paper, an effective and succinct radio-frequency (RF) grounding technique for class-AB power amplifier (PA) is presented. The proposed technique employs a grounding path, resonant with a capacitor in series at the center of the fundamental and second-order harmonic frequencies, between the critical ground nodes, to ensure a low impedance path. The power loss due to imperfect grounding is then reduced by 2 dB, and the saturated output power and power added efficiency (PAE) are therefore significantly improved. A fully integrated 5.8-GHz PA with the proposed technique is designed and implemented in a 65-nm CMOS process. Measured result shows a saturated output power of 25.4 dBm and a peak PAE of 29.7\%, while with only 2.5 V of supply voltage.
\end{info}


Abstract from \href{https://ieeexplore.ieee.org/document/4456783}{Fully Integrated CMOS Power Amplifier With Efficiency Enhancement at Power Back-Off}, Gang Liu, Peter Haldi, Tsu-Jae King Liu, Fellow, IEEE, and Ali M. Niknejad, Member, IEEE

\begin{info}
	\textbf{Abstract} — This paper presents a new approach for power amplifier design using deep submicron CMOS technologies. A transformer based voltage combiner is proposed to combine power generated from several low-voltage CMOS amplifiers. Unlike other voltage combining transformers, the architecture presented in this paper provides greater flexibility to access and control the individual amplifiers in a voltage combined amplifier. In this work this voltage combining transformer has been utilized to control output power and improve average efficiency at power back-off. This technique does not degrade instantaneous efficiency at peak power and maintains voltage gain with power back-off. A 1.2 V, 2.4 GHz fully integrated CMOS power amplifier prototype was implemented with thin-oxide transistors in a 0.13 m RF-CMOS process to demonstrate the concept. Neither off-chip components nor bondwires are used for output matching. The power amplifier transmits 24 dBm power with 25\% drain efficiency at 1 dB compression point. When driven into saturation, it transmits 27 dBm peak power with 32\% drain efficiency. At power back-off, efficiency is greatly improved in the prototype which employs average efficiency enhancement circuitry.
\end{info}

% why does this transformer have greater flexibility

% neither the off-chip components nor bondwires are used for output matching

The cascode gate is connected to the supply node when the individual amplifier is on, and it is grounded to turn off the individual amplifier for power back-off (to improve PAE when the output power is backed off).\\


Abstract from \href{https://ieeexplore.ieee.org/document/4494655}{A 5.8 GHz 1 V Linear Power Amplifier Using a Novel On-Chip Transformer Power Combiner in Standard 90 nm CMOS}

\begin{info}
	\textbf{Abstract} — A fully integrated 5.8 GHz Class AB linear power amplifier (PA) in a standard 90 nm CMOS process using thin oxide transistors utilizes a novel on-chip transformer power combining network. The transformer combines the power of four push-pull stages with low insertion loss over the bandwidth of interest and is compatible with standard CMOS process without any additional analog or RF enhancements. With a 1 V power supply, the PA achieves 24.3 dBm maximum output power at a peak drain efficiency of 27\% and 20.5 dBm output power at the 1 dB compression point
\end{info}

% It does not mention the broadband 

% https://sci-hub.se/10.1007/s11432-016-0299-4

% https://sci-hub.se/10.1109/RWS.2014.6830155 

% PAE is closely connected to operating frequency of power amplifier.

%====================================================================================================================================================================================
\section{Conversation about design}
%====================================================================================================================================================================================

You were talking about previous design and also about need to do redesign. What exact parameters you did not match last time, and do you have also starting design and schematic ready, or you would like to ask us to start completely form beginning?

\textcolor{blue}{[John]: Compared with current design, we would lower the PA output power from max 16dBm to 0dBm, and reduce the PA area as much as possible. We would like to redesign the existing PA because the current design is silicon proven and we have delivered more than 1 million pcs. And we have a redesign strategy in our mind. We can discuss it in the intake meeting.}

I am guessing that we are doing also design and layout due to the fact that it is a RF component?

\textcolor{blue}{[John]:  yes, that is the plan that  you redesign PA and do the layout. We will integrate to the SoC }

I am guessing this component will be further integrated in the complex SOC? 

\textcolor{blue}{[John]: yes, it will be integrated into WiFi SoC. }

Do you have power control of PA?  If yes do you have other specs? 

\textcolor{blue}{[John]: yes, we can propose other specs. However, we would purposely only define the most important specs(EVM, OIP3, VDD, Freq, BW, process technology ) in our design request because we would give the freedom to the PA designer as much as we can. We can discuss  in the intake meeting. }

Do you have any special circutry like power detector to measure if antenna is working (connection ok), or the whole power is reflected back. Self performance test.

\textcolor{blue}{[John]: No, power detector isn't included in this PA redesign work.  } 

Are those parameters down only one for acceptance criteria?

\textcolor{blue}{[John]: They are the key specs. We definitely need more items as acceptance criteria, but we are open to define together with PA design team to give freedom to the PA to make a better trade-off.  }

You were talking about size? What size you have currently?

\textcolor{blue}{[John]: One PA block consists of four identical units. each unit is about 0.13mm2. so for one PA, it is about 0.52mm2.   }

I am guessing we are using our Cadence tools, and simulators?

\textcolor{blue}{[John]: It depends on availability of the CAD tool\&PDK version, the cost and the team preference. we can also align in the intake meeting.} 

%====================================================================================================================================================================================
%====================================================================================================================================================================================

\section{AI help - collected important info}

Technology: TSMC 28nm \\
Supply voltage: 3.3V \\
Frequency range: from 4.9GHz to 5.9GHz \\
output power range: -25dBm to 0dBm with EVM below -38dB. \\
OIP3: 35dBm @ 0dBm \\ % - 10 to 12 dB to get OP1dB 
area: 0.52 mm$^2$

\section{Mail reply and preparation for first meeting}

I assume that these 4 identical power amplifiers are connected as power combined amplifiers (connected in parallel or in series when looked from the output balun). If they are, how aggressively do they want to reduce the area, to throw out PA units, how many would be left in the end? How are outputs of these amplifiers connected? % check what types of connections are in Grujić phd and the main paper

% balun type or like a Wilkinsom that is in the phd.

% power distributed and power combined amplifiers are not the same thing

% how precise is the connection between OP1dB and OIP3

And is there going to be some relaxation for OIP3 from this existing design or does it stay the same?

My additional question is what is the input power for this power amplifier? And does it have any output signal control inside it or does it happen before it?

\newpage

\subsection{How to introduce yourself}

\begin{info}
	The secret is using a simple framework: Present, past, and future.\\
	
	\textbf{Present:} Start with a present-tense statement to introduce yourself: \\
	Hi, I'm Ashley and I'm a software engineer. My current focus is optimizing customer experience.\\
	Nice to meet you all. My name is Michael and I'm the creative director. I work in the Brooklyn office.\\
	Of course, what you share will depend on the situation and on the audience. If you are not sure what to share, your name and job title is a great place to start. If there's an opportunity to elaborate, you can also share other details such as a current project, your expertise, or your geographical location. \\ % probably not geographical location
	
	\textbf{Past:} The second part of your introduction is past tense. This is where you can add two or three points that will provide people with relevant details about your background. It is also your opportunity to establish credibility. Consider your education and other credentials, past projects, employers, and accomplishments.\\
	My background is in computer science. Before joining this team, I worked with big data to identify insights for our clients in the health care industry.\\
	I’ve been at the firm for eight years. Most recently, I worked on the Alpha Financial account, where last year’s campaign won us a Webby award. \\
	
	\textbf{Future:} The third and last part in this framework is future-oriented. This is your opportunity to demonstrate enthusiasm for what’s ahead. If you’re in a job interview, you could share your eagerness about opportunities at the firm. If you’re in a meeting, you could express interest in the meeting topic. If you’re kicking off a project with a new team, you could talk about how excited you are, or share your goals for the project.\\
	I’m honored to be here. This project is a significant opportunity for all of us. \\
	I’m excited to work with you all to solve our clients’ biggest challenges! \\

	That’s it for the self-introduction framework. Present, past, future. Eloquent and effective. By using this approach, you’ll not only introduce yourself better, but it also frees you from ruminating on what you’ll say when it’s your turn to introduce yourself and allows you to listen when others introduce themselves. You will also make it easy for the person who introduces themselves after you, since you’ll conclude your self-introduction with positive enthusiasm.\\ % don't know about enthusiasm
	\href{https://hbr.org/2022/08/a-simple-way-to-introduce-yourself}{\textbf{source}}
\end{info}

My example:\\
\textbf{Present}: Hi, I’m Aleksandar. I am a RF and analog circuit design engineer. \\
\textbf{Past}: I have been working for this company for 5 years.\\
\textbf{Future}: If you have any questions?

\subsection{Anticipated questions}

\begin{question}[\itshape What experience with power amplifiers do you have?]
	I have experience with buffers for LO distribution, buffers and active power dividers. 
	Also I have experience with RF circuits like VCO.
\end{question}


\section{Reading books and papers on the matter}


\subsection{Load-pull contours}

The load-pull method of RF and microwave power amplifier design, John F. Sevic, Ph.D.

% Optimum Impedance Identification

% A Theory for the Prediction of GaAs FET Load-Pull Power Contours, S. Cripps, 1983
% predicting the load-pull counturs
% manually using mechanical tuners is a very laborious operation
% so computer aided test by Cripps is presented
% Load-pull counturs are not circles because of the non linearity of the power amplifier driven into saturation.
% 

%


\subsection{Shive wave machine}

The Shive wave machine, a clever instrument that uses torsion on a wire and suspended weights to propagate a wave that is easy to observe. This machine will help you visualize:
\begin{itemize}
	\item Transmission lines
	\item Propagation velocity
	\item Wavelength
	\item Impedance
	\item Matched loads
	\item Reflections from short and open circuits
	\item Standing waves
	\item Standing wave ratio
	\item Reflection coefficient
	\item Resonance and tuning
	\item Quarter-wave matching networks
	\item Tapered transformers
\end{itemize}



% \begin{table}[ht]
% 	\centering
% 	\begin{tabular}{|c|l|c|c|c|c|c|}
% 		\hline
% 		& LO Requirements & Note & min & typ & max & Units \\
% 		\hline
% 		& \multicolumn{6}{|c|}{VCO Requirements} \\
% 		\hline
% 		1 & Full LO range &  & 6300  &  & 13700 & MHz \\ 
% 		\hline
% 		2 & Phase Noise at 300 kHz &  &  &  & <-101 & dBc/Hz \\ 
% 		\hline
% 		2* & Phase Noise at 1 MHz &  &  &  & <-112.4 & dBc/Hz \\ 
% 		\hline
% 		3 & Vtune &  & 0.1 &  & 1 & V  \\ 
% 		\hline
% 		4 & Tuning Sensitivity $K_{VCO}$ &  &  &  & 100 & MHz/V  \\ 
% 		\hline
% 		5 & Pushing & TBD &  &  & 2 & MHz/V  \\ 
% 		\hline
% 		6 & Output Voltage & TBD & 0.8 &  & & $V_{p-p}$  \\ 
% 		\hline
% 		7 & Load Impedance &  &  &  & 100 & fF  \\ 
% 		\hline
% 		& \multicolumn{6}{|c|}{VCO and output Buffer} \\
% 		\hline
% 		8 & Output Voltage & TBD & 0.8  &  &  & V  \\ 
% 		\hline
% 		9 & Load Impedance & TBD &  &  & 1000 & fF  \\ 
% 		\hline
% 		10 & Harmonic suppression ($2_{nd}$, typ) &  & -15 &  &  & dBc  \\ 
% 		\hline
% 		11 & Pulling (14 dB Return Loss, Any Phase) & TBD &  &  & 2 & MHz  \\ 
% 		\hline
% 		& \multicolumn{6}{|c|}{General Specifications} \\
% 		\hline
% 		12 & Operating Temperature Range &  & -40 &  & 125 & °C  \\ 
% 		\hline
% 		13 & Supply Voltage &  & 1 & 1.1 & 1.2 & V  \\ 
% 		\hline
% 		14 & Supply Current &  &  &  & 20 & mA  \\ 
% 		\hline
% 		15 & Shutdown Current &  &  &  & 10 & $\mu$A  \\ 
% 		\hline
% 		16 & Time to Switch Between Cores  &  &  & 3 & 5 & ms  \\ 
% 		\hline
% 	\end{tabular}
% 	\label{table-spec}
% 	\caption{Specification Requirements} 

% \end{table}

% Min and max for supply voltage are not defined in the original document, and time for switching between two cores was defined during the meeting. Updated 2* specification for phase noise, frequency pushing update not clear.

% Currently Full LO range is split between two cores so the requirement should look like this:

% \begin{table}[ht]
% 	\centering
% 	\begin{tabular}{|c|l|c|c|c|c|c|}
% 		\hline
% 		& LO range requirements & Note & min & typ & max & Units \\
% 		\hline
% 		1 & High Band core LO range &  & 10000  &  & 13700 & MHz \\ 
% 		\hline
% 		2 & Low Band core LO range &  & 6300 &  & 10000 & MHz \\ 
% 		\hline
% 	\end{tabular}
% 	\caption{LO Range Two Core Specification Requirements} 
% \end{table}

% \begin{question}[\itshape What's the variation for supply voltage?]
% 	Not defined probably will be the same as the rest of the 
% \end{question}


% \begin{question}[\itshape $K_{VCO}$ Why is it in the max column? ]
% 	Didn't get answer for this. It's probably also typical and a value that is expected by the PLL design.
% \end{question}

% % Footnotes do not work.

% Process corners can be found at:

% \begin{verbatim}
% /tech/tsmc/tsmc40/models/spectre/crn40lp_2d5_v2d0_2_shrink0d9_embedded_usage.scs
% \end{verbatim}
% % Seems that 

% \section{Scaldio Design Review}

% \begin{question}[\itshape How should calibration be implemented to achieve output voltage peak to peak and minimize noise?]
% Is it done for whole PLL?
% \end{question}

% \begin{question}[\itshape Difference between class-B and class-C vco? ]
% 	Scaldio uses class-C, so all parasitic capacitances connected to the tail don't matter. Having trouble with observing the currents of drain so cannot check this and compare them. Scaldio looks something between class C and class B because of $V_{gbias}$ voltage and because adding inductor between tail NMOS and switching pair improves phase noise. This is attributed to class B, while class C only needs tail capacitance.
% \end{question}




% % \begin{figure}[ht!]
% % 	\includegraphics[width=\linewidth]{Figures/class_C_vs_class_B.png}
% % 	% \caption{Difference between class C and class B}
% % 	\label{fig:classC_classB}
% % \end{figure}

% \begin{figure}[ht!]
% 	\centering
% 	\includegraphics[width=0.7\linewidth]{Figures/class-B_vs_class-C.pdf}
% 	\caption{Schematic difference between class C and class B}
% 	\label{fig:classC_classB_inkscape}
% \end{figure}


% Lowest bit of digital varactor (L0\_PLL\_hbVCO\_Cdig\_SC2B\_VCOS\_SC2C\_PLL) doesn't do anything, isn't even monotonous. Maybe makes more sense when simulating extracted cells. Digital varactor needs to be redesigned, it could lower the Q factor.
% Main difference between class B and class C can be observed by investigating their drain currents.

% \begin{figure}[ht!]
% 	\includegraphics[width=\linewidth]{Figures/drainCurrent_classB_vs_classC.png}
% 	% \caption{Difference between class C and class B}
% 	\label{fig:drainCurrent_classB_vs_classC}
% \end{figure}

% \subsection{Main testbench}

% Main testbench covers everything except for frequency pulling and full LO range. The simulation results are obtained at the higher end of the LO range. Phase noise is simulated by pss+pnoise.

% \begin{info} % Information block
% 	With Noise Type=timeaverage and ALL(AM,PM,USB,LSB), you can plot the AM and PM components as well as the total noise. In addition, you can plot phase noise and FM jitter results for oscillators. Plotting is done using the Direct Plot Form.
% 	\href{https://community.cadence.com/cadence_blogs_8/b/rf/posts/virtuoso-video-diary-noise-simulation-in-spectre-rf-using-improved-pnoise-hbnoise-and-direct-plot-form-options}{\textbf{External link}}
% \end{info}

% Function of phase noise is simulated for PM noise type.

% How to choose beat frequency for autonomous system from forum \href{https://community.cadence.com/cadence_technology_forums/f/custom-ic-design/2661/beat-frequency-in-spectrerf-pss-simulation}{\textbf{thread}}.

% \begin{info} % Information block
% In an autonomous system (e.g. an oscillator), you turn on the "oscillator" checkbox, and the beat frequency is then the estimated frequency, which gives PSS a starting point to solve for the oscillator frequency. It's important when in oscillator mode to select the outputs of the circuit, which include any subharmonics. In other words, if you have an oscillator followed by a divider, point at the divider output, and give the estimated divided frequency as the beat frequency. Again, this is because you need to solve an integer number of cycles of all the frequencies in the circuit. Note, don't use oscillator mode for circuits which aren't oscillators, since you're then trying to get the simulator to solve for an unknown which is not unknown, which may lead to convergence problems.
% \end{info}

% Most of the $I_{bias}$ tune digital control is not used for the higher band, so by increasing the number of steps to cover even higher frequencies than the original design finer bias control is needed.


% \subsection{Simulation Results for Scaldio design}

% Simulated only nominal corner with change for $V_{DD}$ only for frequency pushing simulation. 

% \begin{table}[ht]
% 	\centering
% 	\begin{tabular}{|c|l|c|c|c|c|c|c|}
% 		\hline
% 		& LO Requirements & Note & min & typ & max & Sim(Typ) & Units \\
% 		% \hline
% 		% & \multicolumn{7}{|c|}{VCO Requirements} \\
% 		\hline
% 		1 & Phase Noise at 300 kHz &  &  &  & <-101 & -98 & dBc/Hz  \\ 
% 		\hline
% 		2 & Tuning Sensitivity $K_{VCO}$ &  &  &  & 100 & over & MHz/V  \\ 
% 		\hline
% 		3 & Pushing & TBD &  &  & 2  & 279.7 & MHz/V  \\ 
% 		\hline
% 		4 & Output Voltage & TBD & 800 &  & & 809.3 & $mV_{p-p}$  \\ 
% 		\hline
% 		5 & Harmonic suppression ($2_{nd}$, typ) &  & -15 &  & & -26.78 & dBc  \\ 
% 		\hline
% 		6 & Pulling (14 dB Return Loss, Any Phase) & TBD &  &  & 2  & 19.51 & MHz  \\ 
% 		\hline
% 	\end{tabular}
% 	\label{table-ScaldioResults}
% 	\caption{Scaldio IMEC design Results}
% \end{table}

% Phase noise changes a lot for different tuning voltages between -90 and -100 dBc/Hz. Phase noise is probably not modeled ok because the VCO doesn't have the connection between current bias tail and the switching pair of VCO as transmission line. That inductance and $C_{tail}$ should resonate at 2$\omega_O$ (tank oscillating frequency), but only in the case of class B oscillator. Check what type is vco oscillator. %This is hard to check because pss\_tran drain currents look wierd.

% \newpage

% %----------------------------------------------------------------------------------------
% \section{Q factor of LC tank}

% Need testbenches capacitance of tank, varactors and inductor, and for different controls and also corners. Process corners for varactors are the same as MOSFET.

% % and of tail (to see how to make it resonate at the double of the oscillating frequency).

% \begin{question}[\itshape What kind of chip is it?]
% 	What kind of inductance is expected to be connected on $V_{DD}$ and $V_{SS}$ pins. This may or may not change the results. Removed all together right now.
% \end{question}

% Testbench for differential Q and L of the inductor that is EMX simulated is shown in Figure \ref{fig:qlinductor}.

% \begin{figure}[ht!]
% 	\includegraphics[width=\linewidth]{Figures/QL_inductor.png}
% 	\caption{Q factor and inductance of L}
% 	\label{fig:qlinductor}
% \end{figure}

% Q factor is better at the higher frequency.

% Simulated Q factor of both varactors and inductor. Q factor of digital varactor for lower bit controls is not much better than Q of inductor.

% Capacitor calculating capacitance and Q factor:

% \begin{equation}
% 	Y_{diff} (im) = imag(ypm('sp 1 1)) + imag(ypm('sp 2 2)) - imag(ypm('sp 2 1)) - imag(ypm('sp 1 2))
% \end{equation}

% \begin{equation}
% 	Y_{diff} (re) = real(ypm('sp 1 1)) + real(ypm('sp 2 2)) - real(ypm('sp 2 1)) - real(ypm('sp 1 2))
% \end{equation}

% Capacitance:
% \begin{equation}
% 	C_{diff} = \dfrac{Y_{diff} (im)}{2\pi f}
% \end{equation}

% Q factor of capacitor:
% \begin{equation}
% 	Q_C = \dfrac{Y_{diff} (im)}{ Y_{diff} (re)}
% \end{equation}

% \begin{figure}[h!]
% 	\includegraphics[width=\linewidth]{Figures/Dvaractor.png}
% 	\caption{C and Q of digital varactor through controls}
% 	\label{fig:dvaracator}
% \end{figure}

% More info can be found in  \href{https://ieeexplore.ieee.org/abstract/document/5537949}{\textbf{paper }} A Thorough Analysis of the Tank Quality Factor in LC Oscillators with Switched Capacitor Banks.

% How to calculate Q factor of tank from the same paper.

% \begin{info}
% It can be proved (and it is a well known result) that the quality factor of the tank is equal to the parallel combination of the quality factor of the inductance, $Q_L$, and the quality factor of the capacitance, $Q_C$ , that make up the resonant circuit. Once we have the two quality factors $Q_L$ and $Q_C$ is hence very easy to calculate $Q_T$ .
% \end{info}


% Found definitions:

% \begin{equation}
% 	Q_{LC} = \dfrac{1}{R}\sqrt{\dfrac{L}{C}} = \frac{f_r}{\Delta f} = \frac{\omega _r}{\Delta \omega} = \dfrac{\tau _d \omega}{2}
% \end{equation}

% where $\tau _d$ is group delay. Also


% \begin{equation}
% 	Q_{T} = Q_{LC} = \omega \dfrac{ES}{APD}
% \end{equation}

% where ES is energy stored and APD is average power dissipation.

% \begin{question}[\itshape How to calculate group delay using sparam analyis?]
% 	?
% \end{question}

% \subsection{Calculating Q factor‚ LC tank by ringing method}

% Link to \href{https://www.giangrandi.ch/electronics/ringdownq/ringdownq.shtml}{\textbf{website}} that shows ring down method of calculating Q factor. Not implemented in virtuoso TODO.


% %----------------------------------------------------------------------------------------
% \section{Figure of Merit different definitions}

% \begin{info} % Information block
% 	The theoretical maxima for the FoM of an oscillator is given by $FoM=174+20log_{10}(Q)$ dBc/Hz
% \end{info}

% How to calculate oscillator FoM? Is $Q_L$ factor dominant enough to just equate it with LC tank Q facor.

% The widely used FOM is calculated by:
% \begin{equation}
% 	FoM = L\{ \Delta \omega \}( \Delta f/f_0)^2 PVCO [mW]
% \end{equation}

% Where $L\{\Delta \omega\}$ is the phase noise, $ \Delta \frac{f}{f_0}$ is the ratio between the offset frequency and the carrier, and $P_{VCO}$ is the power consumption of the VCO-core. There is also $FoM^T$ and $FoM_A$ 

% \begin{equation}
% 	FoM^T = FoM - 20 \log (\dfrac{FTR}{10})
% \end{equation}

% and 

% \begin{equation}
% 	FoM_A = FoM + 10 \log (A)
% \end{equation}

% where A is area [$mm^2$], and FTR frequency tuning range [$\%$]

% Information about digital varactor copied from \href{https://www.doe.carleton.ca/~ddchen/Tutorials/DCO.pdf}{\textbf{External link}}

% \begin{info} % Information block
% Each bit of MIM varactor contains two MIM capacitors connected differentially with a series switch, two pull-up and two pull-down transistors to effectively turn the varactor between its high and low-capacitance states. Measured intrinsic Q of the MIM capacitor is 80 at 3.6 GHz. When is turned on, i.e., high-capacitance state, the varactor Q drops to 30. When is turned off to be in low-capacitance state, the parasitic capacitance of the MIM capacitor and transistors has an effective of 50. The pull-down transistors set the DC levels for drain and source of at 0 V so that
% can be efficiently turned into triode region while the weak pull-up transistors set the DC level to VDDOSC to reduce the parasitic capacitance of thus increasing of the parasitic capacitance. The pull-up pMOS can be implemented by either resistors or transistors. The latter was chosen for silicon area efficiency. Compared to MOS varactors, MIM varactors have a much lower . However, since the differential phase-stability inductor is only 10, the impact of lower varactor is tol erable. When the MIM varactor is at its low-capacitance state, the large DCO internal signal swing and the DC level of 1.4-V supply voltage at source/drain of the pull-up transistors force the drain -nwell junction diodes of the pull-up pMOS to momentarily go into forward-bias condition resulting in a latch-up concern. However, since the forward-bias condition occurs only in 50\% of a 3–4 GHz period, the latch-up phenomenon with the parasitic BJTs can not be triggered.
% \end{info}

% %----------------------------------------------------------------------------------------
% \section{Lowering frequency pushing and LDO}

% About Frequency pushing from \href{https://www.atlantis-press.com/article/6376.pdf}{\textbf{this paper}}

% \begin{info} % Information block
% An LC-tank VCO circuit has been implemented in a standard 0.35 $\mu m$ CMOS technology. It is based on a two-transistor biasing structure that improves the performance of frequency pushing and frequency tuning range. Final measurement of proposed structure gives 516 MHz tuning range with 2.278 GHz center frequency and about 0.55$\%$/V frequency pushing in the worst case. The achieved FOM is about -180dBc/Hz, which is very close to the simulated value. This structure is proven to be particularly suitable for achieving low FOM in the VCO circuits having low Q factor LC-tank. Both, the proposed structure and the FOM optimization method, can also be applied to the VCO designs for the applications at higher frequencies, such as 5GHz VCOs for Wireless LAN applications.
% \end{info}

% \begin{question}[\itshape Can VCO work for lower voltage of 0.9 V?]
% 	This may be needed if LDO is required because of frequency pushing?
% \end{question}

% \begin{question}[\itshape Does frequency only happen because of the ripple directly induced by buffers e.g.?]
% 	Or could it happen because of EM crosstalk?
% \end{question}


% Is frequency pushing testbench good? Look into this paper. % which paper
% Different testbench would be to make a transient change in Vdd and check how much frequency changes.


% Results for class C with pmos bias, only dc change of $V_{DD}$:

% \begin{center}
% 	\begin{tabular}{|l|c|c|c|c|c|c|c|c|c|}
% 		\hline
% 		Parameter & typical & spec  & min & max & ss -40 & ss 125 & ff -40 & ff 125 & Units \\
% 		% \hline
% 		% & \multicolumn{7}{|c|}{VCO Requirements} \\
% 		\hline
% 		Frequency Pushing & 43.93 & < 2 &  43.93 & 100.1 & 100.1 & 64.36 & 75.48 & 64.09 & MHz \\ 
% 		\hline
% 		Frequency  & 14.6 & > 14.2  & 14.6 & 15.83 & 15.54 & 15.48 & 15.83 & 15.59 & GHz  \\ 
% 		\hline
% 	\end{tabular}
% \end{center}

% This shows some improvement from frequency pushing of 250 - 300 MHz that was observed for the NMOS bias of IMEC Scaldio design.

% %----------------------------------------------------------------------------------------
% \section{Changing topology and lowering phase noise}

% Decision on why is single sided oscillator chosen instead of double sided oscillator.

% \begin{info}
% However, if non-negligible parasitic capacitances are found at the tank outputs, the phase-noise performance of the DS-VCO may be seriously degraded, while that of the SS-VCO remains unaffected.
% \end{info}

% More on the $\frac{1}{f^2}$ Phase Noise Performance of CMOS Differential-Pair LC-Tank Oscillators in \href{https://backend.orbit.dtu.dk/ws/files/3913656/Andreani.pdf}{\textbf{paper}} by Pietro Andreani.


% \subsection{Hybrid class C and class B}

% \begin{info}
% Further, the proposed VCO solves the issue of the hybrid mixed-signal start-up procedure exposed in [8]. The main drawback of this approach is that, if oscillation stops for some unaccountable reason, the VCO can only be restarted actuating again the whole start-up procedure.
% \end{info}

% Referenced paper is:

% \begin{itemize}
% 	\item [8] J. Chen, F. Jonsson, M. Carlsson, C. Hedenas, and L.-R. Zheng, “\href{https://ieeexplore.ieee.org/document/5951800}{A low power, startup ensured and constant amplitude class-C VCO in \SI{0.18}{\micro\metre} CMOS},” IEEE Microw. Wireless Compon. Lett., vol. 21, no. 8, pp. 427–429, 2011. Dec. 2008.
% \end{itemize}


% \subsection{Enhanced Oscillation Swing}

% Class-C VCO With Amplitude Feedback Loop for Robust Start-Up and Enhanced Oscillation Swing in \href{https://ieeexplore.ieee.org/stamp/stamp.jsp?arnumber=6377236}{\textbf{this paper.}} Phase noise is lowered by lowering $V_{gbias}$, but it makes oscillations start up harder.

% \begin{info}
% As noted above, the phase noise improves with increasing the oscillation amplitude, which here would mean lowering the gate bias voltage, $V_{bias}$ . Unfortunately, the original class-C oscillator limits the fixed $V_{bias}$ from being set low enough, otherwise the oscillation may not start up. In [11], a high-swing class-C (HSCC) oscillator was introduced, which removed the tail current transistor of the original class-C oscillator [6]. Instead, an automatic amplitude control was introduced to stabilize the oscillation amplitude. In this work, instead of the transformer used in [11], we choose a simple RC bias circuit.
% \end{info}

% This is from \href{https://www.semanticscholar.org/paper/Dual-Core-High-Swing-Class-C-VCO-design-Kim-Kim/c9551af0809604f76263af49976df9efc213bb8e}{\textbf{paper}} Dual-Core High-Swing Class-C VCO design, and references 


% \begin{itemize}
% 	\item [6] A. Mazzanti and P. Andreani, “\href{https://ieeexplore.ieee.org/document/4684621}{Class-C harmonic CMOS
% 	VCOs, with a general result on phase noise},” IEEE J.
% 	Solid-State Circuits, vol. 43, no. 12, pp. 2716–2729,
% 	Dec. 2008.
% 	\item [11] M. Tohidian, A. Fotowat-Ahmadi, M. Kamarei, and F.
% 	Ndagijimana, “\href{https://ieeexplore.ieee.org/document/6045015}{High-swing class-C VCO},” in Proc.
% 	ESSCIRC, Sep. 2011, pp. 495–498
% \end{itemize}


% \subsection{Questions about new double feedback}


% Amplitude Control Feedback and Voltage bias control for Robust start up. 

% % \begin{question}[\itshape What is?]
% % 	This may be needed if LDO is required because of frequency pushing?
% % \end{question}


% \subsection{OTA1 - Start up  and gate voltage bias control}

% Referent voltage should be set and controlled around 800 mV. Need tests for OTAs inside for VCO, currently simulated only PM and DC gain, should test different $V_{ref}$ levels.

% \subsection{OTA2 - Start up and bias control}

% Referent voltage should be set and controlled around 400 mV. Problem with OTA2 loop is amplifying noise into the tail bias current. So the new problem arises as noise shaping in LOOP2 is needed. If a OTA of low uGBW is used than start up is too slow.


% %----------------------------------------------------------------------------------------
% \section{Full LO Range and Frequency Recentering}

% % TODO Setup the lb vco, and also setup different testbenches for the highest and the lowest frequency available. Add corner analysis PVT.

% TODO Add corners for \textbf{fs sf} and similar. Because the LO range is split on two cores, ideally halved. Results are simulated for process and temperature corners: 

% \begin{table}[ht]
% 	\centering
% 	\begin{tabular}{|l|c|c|c|c|c|}
% 		\hline
% 		Core and Specifictaion & min & max & sim(min) & sim(max) & Units \\
% 		\hline
% 		\multicolumn{6}{|c|}{VCO core split} \\
% 		\hline
% 		High Band Higher limit & 13700 &  & 15580 & 16190  &  MHz  \\ 
% 		\hline
% 		High Band Lower limit &  & 10000 & 10870 & 12460 &  MHz  \\ 
% 		\hline
% 		High Band range & 3700 &  & 4710 & 3730 &  MHz  \\ 
% 		\hline
% 		Low Band Higher limit & 10000 &   & 13230 & 14120 &  MHz  \\ 
% 		\hline
% 		Low Band Lower limit &  & 6300 & 6724 & 8028  &  MHz  \\ 
% 		\hline
% 		Low Band range & 3700 &  & 6506 & 6092 &  MHz  \\ 
% 		\hline
% 	\end{tabular}
% 	\caption{LO range specification and simulation for Scaldio LC tank}
% \end{table}

% Expecting the drop for schematic simulated frequency range, covered range should at least be larger than needed when recentered: 

% \begin{equation}
% 	HBFR = 13700 - 10000 = 3700 < 3730 = 16190 - 12460
% \end{equation}

% For lower band it's similarily calculated

% \begin{equation}
% 	HBFR = 10000 - 6300 = 3700 < 6092 = 14120 - 8028
% \end{equation}

% Lower and higher band are assymetrical and they overlap for at least:

% \begin{equation}
% 	HBLBoverlap_{high} = 13230 - 10870 = 2360
% \end{equation}

% \begin{equation}
% 	HBLBoverlap_{low} = 14120 - 12460 = 1660
% \end{equation}

% The worst available frequency digital controlled range is $6092 + 3730 - 1660 = 8162$ MHz which is higher than 7400 MHz.

% \begin{question}[\itshape Are the two VCO-s in the same process corner at the same time?]
% 	Yes. If not than the calculations are wrong.
% \end{question}

% NOTE: Analog varactor 0-Vt-1 change does not work so it wasn't included. Further frequency recentering after extraction will be needed.

% \subsection{Frequency Tuning Range (FTR)}

% Frequency tuning range is calculated:
% %----------------------------------------------------------------------

% UHF channel radio 380 MHz - 512 MHz. TOCHECK
% Why does it matter for the phase noise.

% %----------------------------------------------------------------------------------------
% \section{Pulling testbench and design of VCO buffer}

% Needs port at and tuning circuit to keep the reflection at -14 dB. Port of reference impedance 10 k$\Omega$ and portAdapter from rfExamples. S parameter analysis to show if the reflection really is -14 dB? Load impedance increased from 100 fF to 1000 fF. VCO buffer is also needed because of the frequency pulling. Does each core have a buffer or do they share it? first make a buffer than maybe a question.

% By adding two buffers from \verb c LO_FDDQ_v6c, \verb c LO_FDDQ_INHB_v5_JCx_scaldio2b  and \verb cLO_FDDQ_BUF_X24 c, frequency pulling drops below 1 MHz. 

% So this specification looks fine, testbench shown:

% \begin{question}[\itshape How does portAdapter work?]
% \end{question}

% Frequency pulling mentions coupling (crosstalk) between different blocks on chip and the VCO.
% %----------------------------------------------------------------------------------------
% \section{Bandgap - PTAT and CTAT, the temperature independant Vbias for cascode}

% AC noise results of the bandgap is shown in Fig. \ref{fig:ac_noise_Bandgap}.

% \begin{figure}[!ht]
% 	\includegraphics[width=\linewidth]{Figures/ac_noise_Bandgap.jpg}
% 	\caption{AC Noise result of bandgap}
% 	\label{fig:ac_noise_Bandgap}
% \end{figure}

% \ref{fig:phase_noise_Bandgap_pBias_LNeq0&1}

% \begin{figure}[!ht]
% 	\includegraphics[width=\linewidth]{Figures/phase_noise_Bandgap_pBias_LNeq0&1.jpg}
% 	\caption{Phase Noise of the Bandgap with PMOS bias when LN is set to 0 or 1}
% 	\label{fig:phase_noise_Bandgap_pBias_LNeq0&1}
% \end{figure}

% TODO Maybe add some results over PVT for Bandgap

% TODO Use this info on the original NMOS VCO Scaldio design.

% \subsection{PTAT and CTAT, the temperature independant Vbias for cascode}

% % \verb c c

% TODO Simulate and show results of temperature sweep. (probably done maybe plot)

% %----------------------------------------------------------------------------------------

% \section{Design of tail transistors}

% Tail transistors need to be designed carefully because of their effect on the phase noise performance of VCO.

% If the inductor is resonating at double the oscillation frequency with the parasitic capacitance at the common nMOS-pair source, this technique greatly reduces both and white noise from the tail bias transistor [12]. It is, however, associated with two momentous drawbacks: it is narrowband in nature

% \subsection{Effect of too high tail capacitance}

% \begin{figure}[!ht]
% 	\includegraphics[width=\linewidth]{Figures/squegging.png}
% 	\caption{Squegging}
% 	\label{fig:squegging}
% \end{figure}

% Instability Low frequency amplitude modulation squegging. This was the issue with the original class C with tail current. Is it important for new topology where current bias is PMOS.

% TODO simulate for high and for low frequency. How to simulate envelope? 


% \section{PLL theory - PLL Lock Detect and calibration}

% About PLL Lock Detect:

% \begin{info} % Information block
% 	The ability for a PLL to reliably indicate when it is in lock is critical for many applications. An ideal lock detect circuit gives a high indication when the PLL is locked and a low indication when the PLL is unlocked. When VCO calibration finishes it can be indicative if the lock is detected. 
% \end{info}

% PLL VCO calibration usually goes as amplitude frequency and then amplitude calibration, or the other way?

% Divide and Conquer algorithm is used to find the lock.

% \begin{figure}[!ht]
% 	\includegraphics[width=\linewidth]{Figures/PLL_basics_regarding_VCO.png}
% 	\caption{Gain for VCO and  other blocks inside of PLL}
% 	\label{fig:PLL_basics_regarding_VCO	}
% \end{figure}

% \begin{info}
% 	For the VCO, the noise is suppressed below the loop bandwidth frequency and unshaped above the loop bandwidth.
% \end{info}

% PLL must implement shutdown mode.

% \subsection{Definition of PLL}

% PLL is an automatic control system that adjusts the frequency /phase of the controlled voltage generator (VCO). The type of frequency division is integer (1/N) + fractional (1/224).  The PLL unit includes:  an integer unit of the reference frequency divider R, a frequency-phase detector PFD and a charge pump CP circuit, integer and fractional dividers N + F of the VCO frequency (fractional circuit controls the divider c sigma-delta modulator with programmable order from 1 to 4. The denominator of the divisor fraction has a bit depth of 24 bits), loop filter Loop Filter.

% \section{FDDQ block}

% FDDQ block is connected to the PLL and VCO output, as VCO output buffer. The block LO25 which is located close to mixers generates 25\% duty cycle.

% \begin{question}[\itshape What is considered VCO buffer ?]
% 	Whole FDDQ or just inhb and inlb?
% \end{question}





% 50 or 100 ns for the SAW filter

% differential instead of pseudo differential 

% source degeneration


% \begin{question}[\itshape What kind of output is needed ?]
% 	s
% \end{question}


% % Math equation/formula
% \begin{equation}
% 	I = \int_{a}^{b} f(x) \; \text{d}x.
% \end{equation}

% \begin{info} % Information block
% 	This is an interesting piece of information, to which the reader should pay special attention. Fusce varius orci ac magna dapibus porttitor. In tempor leo a neque bibendum sollicitudin. Nulla pretium fermentum nisi, eget sodales magna facilisis eu. Praesent aliquet nulla ut bibendum lacinia. Donec vel mauris vulputate, commodo ligula ut, egestas orci. Suspendisse commodo odio sed hendrerit lobortis. Donec finibus eros erat, vel ornare enim mattis et.
% \end{info}

% %----------------------------------------------------------------------------------------
% %	PROBLEM 1
% %----------------------------------------------------------------------------------------

% \section{Problem title} % Numbered section

% In hac habitasse platea dictumst. Curabitur mattis elit sit amet justo luctus vestibulum. In hac habitasse platea dictumst. Pellentesque lobortis justo enim, a condimentum massa tempor eu. Ut quis nulla a quam pretium eleifend nec eu nisl. Nam cursus porttitor eros, sed luctus ligula convallis quis. Nam convallis, ligula in auctor euismod, ligula mauris fringilla tellus, et egestas mauris odio eget diam. Praesent sodales in ipsum eu dictum.

% %------------------------------------------------

% \subsection{Theoretical viewpoint}

% Maecenas consectetur metus at tellus finibus condimentum. Proin arcu lectus, ultrices non tincidunt et, tincidunt ut quam. Integer luctus posuere est, non maximus ante dignissim quis. Nunc a cursus erat. Curabitur suscipit nibh in tincidunt sagittis. Nam malesuada vestibulum quam id gravida. Proin ut dapibus velit. Vestibulum eget quam quis ipsum semper convallis. Duis consectetur nibh ac diam dignissim, id condimentum enim dictum. Nam aliquet ligula eu magna pellentesque, nec sagittis leo lobortis. Aenean tincidunt dignissim egestas. Morbi efficitur risus ante, id tincidunt odio pulvinar vitae.

% Curabitur tempus hendrerit nulla. Donec faucibus lobortis nibh pharetra sagittis. Sed magna sem, posuere eget sem vitae, finibus consequat libero. Cras aliquet sagittis erat ut semper. Aenean vel enim ipsum. Fusce ut felis at eros sagittis bibendum mollis lobortis libero. Donec laoreet nisl vel risus lacinia elementum non nec lacus. Nullam luctus, nulla volutpat ultricies ultrices, quam massa placerat augue, ut fringilla urna lectus nec nibh. Vestibulum efficitur condimentum orci a semper. Pellentesque ut metus pretium lacus maximus semper. Sed tellus augue, consectetur rhoncus eleifend vel, imperdiet nec turpis. Nulla ligula ante, malesuada quis orci a, ultricies blandit elit.

% % Numbered question, with subquestions in an enumerate environment
% \begin{question}
% 	Quisque ullamcorper placerat ipsum. Cras nibh. Morbi vel justo vitae lacus tincidunt ultrices. Lorem ipsum dolor sit amet, consectetuer adipiscing elit.

% 	% Subquestions numbered with letters
% 	\begin{enumerate}[(a)]
% 		\item Do this.
% 		\item Do that.
% 		\item Do something else.
% 	\end{enumerate}
% \end{question}
	
% %------------------------------------------------

% \subsection{Algorithmic issues}

% In malesuada ullamcorper urna, sed dapibus diam sollicitudin non. Donec elit odio, accumsan ac nisl a, tempor imperdiet eros. Donec porta tortor eu risus consequat, a pharetra tortor tristique. Morbi sit amet laoreet erat. Morbi et luctus diam, quis porta ipsum. Quisque libero dolor, suscipit id facilisis eget, sodales volutpat dolor. Nullam vulputate interdum aliquam. Mauris id convallis erat, ut vehicula neque. Sed auctor nibh et elit fringilla, nec ultricies dui sollicitudin. Vestibulum vestibulum luctus metus venenatis facilisis. Suspendisse iaculis augue at vehicula ornare. Sed vel eros ut velit fermentum porttitor sed sed massa. Fusce venenatis, metus a rutrum sagittis, enim ex maximus velit, id semper nisi velit eu purus.

% \begin{center}
% 	\begin{minipage}{0.5\linewidth} % Adjust the minipage width to accomodate for the length of algorithm lines
% 		\begin{algorithm}[H]
% 			\KwIn{$(a, b)$, two floating-point numbers}  % Algorithm inputs
% 			\KwResult{$(c, d)$, such that $a+b = c + d$} % Algorithm outputs/results
% 			\medskip
% 			\If{$\vert b\vert > \vert a\vert$}{
% 				exchange $a$ and $b$ \;
% 			}
% 			$c \leftarrow a + b$ \;
% 			$z \leftarrow c - a$ \;
% 			$d \leftarrow b - z$ \;
% 			{\bf return} $(c,d)$ \;
% 			\caption{\texttt{FastTwoSum}} % Algorithm name
% 			\label{alg:fastTwoSum}   % optional label to refer to
% 		\end{algorithm}
% 	\end{minipage}
% \end{center}

% Fusce varius orci ac magna dapibus porttitor. In tempor leo a neque bibendum sollicitudin. Nulla pretium fermentum nisi, eget sodales magna facilisis eu. Praesent aliquet nulla ut bibendum lacinia. Donec vel mauris vulputate, commodo ligula ut, egestas orci. Suspendisse commodo odio sed hendrerit lobortis. Donec finibus eros erat, vel ornare enim mattis et.

% % Numbered question, with an optional title
% \begin{question}[\itshape (with optional title)]
% 	In congue risus leo, in gravida enim viverra id. Donec eros mauris, bibendum vel dui at, tempor commodo augue. In vel lobortis lacus. Nam ornare ullamcorper mauris vel molestie. Maecenas vehicula ornare turpis, vitae fringilla orci consectetur vel. Nam pulvinar justo nec neque egestas tristique. Donec ac dolor at libero congue varius sed vitae lectus. Donec et tristique nulla, sit amet scelerisque orci. Maecenas a vestibulum lectus, vitae gravida nulla. Proin eget volutpat orci. Morbi eu aliquet turpis. Vivamus molestie urna quis tempor tristique. Proin hendrerit sem nec tempor sollicitudin.
% \end{question}

% Mauris interdum porttitor fringilla. Proin tincidunt sodales leo at ornare. Donec tempus magna non mauris gravida luctus. Cras vitae arcu vitae mauris eleifend scelerisque. Nam sem sapien, vulputate nec felis eu, blandit convallis risus. Pellentesque sollicitudin venenatis tincidunt. In et ipsum libero. Nullam tempor ligula a massa convallis pellentesque.

% %----------------------------------------------------------------------------------------
% %	PROBLEM 2
% %----------------------------------------------------------------------------------------

% \section{Implementation}

% Proin lobortis efficitur dictum. Pellentesque vitae pharetra eros, quis dignissim magna. Sed tellus leo, semper non vestibulum vel, tincidunt eu mi. Aenean pretium ut velit sed facilisis. Ut placerat urna facilisis dolor suscipit vehicula. Ut ut auctor nunc. Nulla non massa eros. Proin rhoncus arcu odio, eu lobortis metus sollicitudin eu. Duis maximus ex dui, id bibendum diam dignissim id. Aliquam quis lorem lorem. Phasellus sagittis aliquet dolor, vulputate cursus dolor convallis vel. Suspendisse eu tellus feugiat, bibendum lectus quis, fermentum nunc. Nunc euismod condimentum magna nec bibendum. Curabitur elementum nibh eu sem cursus, eu aliquam leo rutrum. Sed bibendum augue sit amet pharetra ullamcorper. Aenean congue sit amet tortor vitae feugiat.

% In congue risus leo, in gravida enim viverra id. Donec eros mauris, bibendum vel dui at, tempor commodo augue. In vel lobortis lacus. Nam ornare ullamcorper mauris vel molestie. Maecenas vehicula ornare turpis, vitae fringilla orci consectetur vel. Nam pulvinar justo nec neque egestas tristique. Donec ac dolor at libero congue varius sed vitae lectus. Donec et tristique nulla, sit amet scelerisque orci. Maecenas a vestibulum lectus, vitae gravida nulla. Proin eget volutpat orci. Morbi eu aliquet turpis. Vivamus molestie urna quis tempor tristique. Proin hendrerit sem nec tempor sollicitudin.

% % File contents
% \begin{file}[hello.py]
% \begin{lstlisting}[language=Python]
% #! /usr/bin/python

% import sys
% sys.stdout.write("Hello World!\n")
% \end{lstlisting}
% \end{file}

% Fusce eleifend porttitor arcu, id accumsan elit pharetra eget. Mauris luctus velit sit amet est sodales rhoncus. Donec cursus suscipit justo, sed tristique ipsum fermentum nec. Ut tortor ex, ullamcorper varius congue in, efficitur a tellus. Vivamus ut rutrum nisi. Phasellus sit amet enim efficitur, aliquam nulla id, lacinia mauris. Quisque viverra libero ac magna maximus efficitur. Interdum et malesuada fames ac ante ipsum primis in faucibus. Vestibulum mollis eros in tellus fermentum, vitae tristique justo finibus. Sed quis vehicula nibh. Etiam nulla justo, pellentesque id sapien at, semper aliquam arcu. Integer at commodo arcu. Quisque dapibus ut lacus eget vulputate.

% % Command-line "screenshot"
% \begin{commandline}
% 	\begin{verbatim}
% 		$ chmod +x hello.py
% 		$ ./hello.py

% 		Hello World!
% 	\end{verbatim}
% \end{commandline}

% Vestibulum sodales orci a nisi interdum tristique. In dictum vehicula dui, eget bibendum purus elementum eu. Pellentesque lobortis mattis mauris, non feugiat dolor vulputate a. Cras porttitor dapibus lacus at pulvinar. Praesent eu nunc et libero porttitor malesuada tempus quis massa. Aenean cursus ipsum a velit ultricies sagittis. Sed non leo ullamcorper, suscipit massa ut, pulvinar erat. Aliquam erat volutpat. Nulla non lacus vitae mi placerat tincidunt et ac diam. Aliquam tincidunt augue sem, ut vestibulum est volutpat eget. Suspendisse potenti. Integer condimentum, risus nec maximus elementum, lacus purus porta arcu, at ultrices diam nisl eget urna. Curabitur sollicitudin diam quis sollicitudin varius. Ut porta erat ornare laoreet euismod. In tincidunt purus dui, nec egestas dui convallis non. In vestibulum ipsum in dictum scelerisque.

% % Warning text, with a custom title
% \begin{warn}[Notice:]
%   In congue risus leo, in gravida enim viverra id. Donec eros mauris, bibendum vel dui at, tempor commodo augue. In vel lobortis lacus. Nam ornare ullamcorper mauris vel molestie. Maecenas vehicula ornare turpis, vitae fringilla orci consectetur vel. Nam pulvinar justo nec neque egestas tristique. Donec ac dolor at libero congue varius sed vitae lectus. Donec et tristique nulla, sit amet scelerisque orci. Maecenas a vestibulum lectus, vitae gravida nulla. Proin eget volutpat orci. Morbi eu aliquet turpis. Vivamus molestie urna quis tempor tristique. Proin hendrerit sem nec tempor sollicitudin.
% \end{warn}

% %----------------------------------------------------------------------------------------

\end{document}
